\documentclass[a4paper]{memoir}
% This file was generated by layout1 for html2latex1, which is free software licensed under the GNU Affero General Public License 3 or any later version (see https://github.com/publishing-systems/automated_digital_publishing/ and http://www.publishing-systems.org).

\usepackage[utf8]{inputenc}
\usepackage[T1]{fontenc}
\usepackage{lmodern}
\usepackage{ngerman}
\usepackage{url}

\setlength{\parskip}{0pt}

\renewcommand*{\maketitle}{%
  \thispagestyle{empty}
  \begingroup%
    \vspace*{\baselineskip}
    \vfill
    \hbox{%
      \vbox{%
        \vspace{0.1\textheight}
        {\noindent\HUGE\bfseries\centering Kommentare zum Literaturbetrieb in Zeiten der Digitalisierung\par}\vspace*{5\baselineskip}
        \vspace{0.35\textheight} 
      }%
    }%
    \vfill
    \null
  \endgroup%
  \pagebreak{}
  \newpage{}
  \thispagestyle{empty}
  \mbox{}
  \pagebreak{}
}
    
\begin{document}
\frontmatter
\maketitle
\tableofcontents
\mainmatter
\chapter{Werkzeugkiste für Self-Publisher und jedermann}
Für Autoren, die einmal ein einzelnes Buch veröffentlichen wollen, ist die Sache relativ einfach: Programme wie Sigil\footnote{\url{https://github.com/user-none/Sigil/releases}} oder LibreOffice\footnote{\url{https://de.libreoffice.org/}} können dazu verwendet werden, um die üblichen Zielformate EPUB und PDF zu erzeugen. Doch schnell werden die Grenzen deutlich, die sich aus einer allzu manuellen Herangehensweise ergeben, denn nachträgliche Korrekturen können umfangreiche Anpassungen erforderlich machen. Erst recht der Umgang mit größeren Textbeständen oder besondere Aufbereitungs-Anforderungen verlangen Software-Lösungen, die wiederkehrende Arbeitsschritte automatisieren und gleichzeitig nicht statisch auf einen einzigen Anwendungsfall festgelegt sind, sondern flexibel im Rahmen unterschiedlicher Projekte eingesetzt werden können – mit anderen Worten: es braucht eine Werkzeugkiste für Self-Publisher, Verlage neuen Typs, Verlage alten Typs und auch sonst jedermann.

Günstigerweise hat die Technik im Bereich der Medienproduktion in den letzten Jahrzehnten große Fortschritte gemacht, was nicht zuletzt auch Standardisierungsprozessen und der Verfügbarkeit von frei lizenzierten Implementierungen zu verdanken ist. Während nun die alte Verlagswelt langsam umsteigt auf „Digital First“-Workflows, darf die Self-Publisher-Szene dahinter nicht zurückbleiben, zumal das Veröffentlichungsmonopol ja gerade infolge der zunehmenden Digitalisierung weggefallen ist und es gilt, das entstandene Vakuum mit einem neuartigen Literaturbetrieb zu besetzen, anstatt un- und antidigitalen Angeboten Vorschub zu leisten. Die Teilnehmer an diesem digitalen Literaturbetrieb benötigen jedoch alle geeignete Werkzeuge und haben teils schon angefangen, sich jeweils eigene Lösungen zu schaffen. Weil man aber niemals um die Technik, sondern um die Bedeutsamkeit von Inhalten konkurrieren sollte, ist es an der Zeit, eine gemeinsame technische Basis zu schaffen, die das unabhängige Publizieren insgesamt fördert, anstatt wieder und wieder in die Lösung bereits gelöster Aufgabenstellungen zu investieren. Natürlich gibt es bereits diverse Projekte, die an einer solchen gemeinsamen Basis oder auch an einzelnen Komponenten dafür arbeiten. Mit publishing-systems.org\footnote{\url{http://www.publishing-systems.org/index.php?lang=de}} sollen diese Bemühungen verzeichnet, bewertet, unterstützt, miteinander verknüpft und ergänzt werden. Es folgt nun eine Beschreibung der Design-Kriterien für unsere eigene Software als auch unseres Projekts an sich.

\section{Projektziele, Design-Kriterien für eigene Software}


Der wichtigste Grundsatz lautet \textbf{freie Lizenzierung}. Seit der allgemeinen Verfügbarkeit von Digitaltechnologie und der darin inherenten Entkoppelung der Information von ihrem physischen Träger bei gleichzeitig immenser Skalierbarkeit durch den Computer als Universalrechenmaschine (Interoperabilität) steht fest, dass das prä-digitale Urheberrecht in seiner gegenwärtigen Form dringend reformbedürftig ist, weil es mittlerweile infolge technischer, rechtlicher, sozialer und ethischer Probleme das genaue Gegenteil von dem erreicht, wofür es eigentlich gedacht war. Seit den 80er-Jahren tobt der Kampf um digitale Grundrechte in den Bereichen Software, Musik, Video und jetzt auch bei den Büchern, wo aber die Öffentlichkeit wenig organisierte Interessensvertretung hervorgebracht hat und der Gesetzgeber tendenziell eher den Vorschlägen restriktiver Rechteverwerter folgt.

Bei Software und manchen Formaten ist eine Form von DRM schon allein aus technischen Gründen gegeben, sodass die unabhängige, eigenständige Datenverarbeitung effektiv unmöglich gemacht werden kann. Die Anwendung des Urheberrechts auf Software trotz ihres Charakters als Werke von praktischem Nutzen tut ihr Übriges, um Nutzer und andere Entwickler in Abhängigkeit zu bringen. Die einzigste Chance besteht im Moment darin, dass die Urheber ihre Software frei lizenzieren, indem Nutzern umfangreiche Nutzungsrechte eingeräumt werden und anderen Entwicklern die kollaborative Zusammenarbeit ermöglicht wird. Darum ist unsere Software unter der GNU Affero General Public License 3\footnote{\url{https://www.gnu.org/licenses/agpl-3.0.html}} or any later version lizenziert, um die Verfügbarkeit, Veränderbarkeit, Weiterverbreitbarkeit und unabhängige Verwendbarkeit langfristig sicherzustellen.

Wenn es um Medienwerke wie Bücher geht, ist freie Lizenzierung genauso wie bei der zugrundeliegenden verarbeitenden Software eine elementare Voraussetzung für einen ganzheitlich digitalverträglichen Literaturbetrieb. Vielfältig muss das digitale Potential im Hinblick auf den Umgang mit Texten unausgeschöpft werden, weil immer noch das Urheberrecht dazu missbraucht wird, das Geschäftsmodell des Verkaufs einzelner Dateikopien als Nachbildung der Knappheit aus der physischen Welt einzuführen, obwohl der digitalen Welt der Überfluss zugrundeliegt und die künstliche Verknappung dem Anliegen und Zweck einer Veröffentlichung direkt entgegensteht („Veröffentlichung“ hat etwas mit Öffentlichkeit zu tun). Ziel muss es deshalb sein, die Arbeit an und mit frei lizenzierten sowie gemeinfreien Texten aktiv voranzutreiben, die dafür vorhandenen Geschäftsmodelle weiter zu etablieren und so etwas wie eine freie digitale Bibliothek in welcher Form auch immer entstehen zu lassen, die in den Genuss umfangreicher als auch vielfältiger Community- und Softwareunterstützung kommen wird.

Mehr zum Thema von Richard Stallman (über Software\footnote{\url{https://archive.org/details/Richard.Stallman.Manchester.2008}}, über andere Medienwerke\footnote{\url{https://www.youtube.com/watch?v=eginMQBWII4}}), Cory Doctorow (über die Implikationen von Digitaltechnologie\footnote{\url{https://www.youtube.com/watch?v=nZFg-uq5zBA}}, über DRM\footnote{\url{https://www.youtube.com/watch?v=HUEvRyemKSg}}) und The League Of Noble Peers (Steal This Film 2\footnote{\url{https://archive.org/details/StealThisFilmII}}).

Die nächste wesentliche Eigenschaft der Software in der Werkzeugkiste betrifft die \textbf{Automatisierung}. Es gibt einen ständig wachsenden Bestand an Texten und jeder dieser Texte bedarf umfangreicher Pflege, Anreicherung und Aufbereitung. Anstatt viel Zeit mit manueller Bearbeitung zu verschwenden, die bei der nächsten Gelegenheit schon wieder hinfällig werden kann, sollen wiederkehrende Arbeitsschritte möglichst automatisiert werden, damit mehr Zeit für automatisierungskompatible Anreicherungen bleibt. Die Automatisierung geht keineswegs zulasten der Qualität oder der Gestaltungsfreiheit, sondern kann im Rahmen der Verarbeitung durchaus Konfigurationseinstellungen oder die Einbindung externer Module zwecks Sonderbehandlung vorsehen, sodass der Workflow ein durchgängig datengetriebener ist und die Automatisierungssoftware lediglich die Methodik zur Verfügung stellt, um von einer definierten Eingabe zur gewünschten Ausgabe zu gelangen, womit der Prozess beliebig oft auf dieselben oder kompatible Eingabedaten neu angewendet werden kann.

Die technische Grundlage dafür stellt XML als universaler Standard zur Auszeichnung von textorientierten Daten dar, womit zahlreiche auf bestimmte Anwendungsbereiche spezialisierte Formate wie z.B. HTML, DocBook, ODT oder TEI definiert wurden, wofür in quasi allen Programmiersprachen Zugriffs- und Verarbeitungsbibliotheken zur Verfügung stehen, worauf mächtige Werkzeuge wie XML-Schema-Validierer, XSLT- und XSL-FO-Prozessoren operieren. XML sichert dabei die Interoperabilität, indem zwischen unterschiedlichen Formaten hin- und herkonvertiert werden kann, Nicht-XML-Quelldateien über Parser nach XML und XML-Daten über entsprechende darauf zugeschnittene Generatoren (evtl. irreversibel) in die gewünschten Zielformate überführt werden. Freilich sind Konzepte wie Single Source Publishing\footnote{\url{https://en.wikipedia.org/wiki/Single_source_publishing}} oder gar „Multi Source Publishing“ (Erzeugen mehrerer Zielformate unter der Zusammenfügung mehrerer Datenquellen, die womöglich ihrerseits jeweils mehrfach konvertiert werden müssen) ganz im Sinne der Automatisierung. Allerdings soll die technische Komplexität der Formate, der Verarbeitung und der Abstimmung der Werkzeuge vor dem Benutzer durch grafische Oberflächen komplett verborgen werden (siehe Robert Cailliau zu den Grundlagen des World Wide Webs\footnote{\url{https://www.youtube.com/watch?v=x2GylLq59rI}}), weil einerseits manuelle Eingriffe beim nächsten Durchlauf hinfällig wären und andererseits die Erstellung qualitativer Quelldokumente und die Definition von Workflows mithilfe von sinnvollen Standardvorgaben stark vereinfacht werden können.

Die Automatisierung profitiert ganz erheblich von der \textbf{Modularität} der Software in der Werkzeugkiste. Die meiste Software für Self-Publisher folgt einer monolithischen Architektur, d.h. ein einziges großes Programm vereint sämtliche angebotene Funktionalität in sich selbst mit dem Ziel, eine integrierte Arbeitsumgebung bereitzustellen. Der Nachteil davon ist, dass einzelne Funktionen oft nicht separat aufgerufen werden können, sondern stattdessen immer die komplette Arbeitsumgebung gestartet werden muss. Dann ist die Arbeitsumgebung in der Regel eine grafische, sodass die Bedienung des Programms primär mit der Maus erfolgt. Bei der Automatisierung macht es aber nur wenig Sinn, Klicks auf Buttons und Menüs auszulösen, denn die Eingabe von Daten soll ja schließlich von außen parametrisiert angesteuert werden, um einen rein datengetriebenen Ablauf zu erreichen. 

Dementsprechend soll die Software, die im Rahmen des Projekts entwickelt wird, aus einer Reihe von kleinen Einzelprogrammen bestehen, die zu größeren automatisierten Workflows zusammengeschaltet werden können. Einerseits entsteht dadurch eine hohe Flexibilität, weil die Programme auch in anderen Kombinationen eingesetzt werden können, andererseits können zwischen den einzelnen Verarbeitungsschritten externe Komponenten aufgerufen werden. Dass die Programme auch abseits automatischer Workflows zur Bewältigung begrenzter Aufgaben in einem ansonsten manuellen Aufbereitungsworkflow genutzt werden können, versteht sich von selbst. Dieser Ansatz schließt übrigens keineswegs aus, dass vor oder während der Verarbeitung umfangreiche benutzerspezifische Einstellungen hinterlegt werden können, die dann von den jeweiligen Einzelschritten berücksichtigt werden. Eine optionale grafische Oberfläche ist dabei behilflich, die überdies auch zur Steuerung und Bedienung der Einzelprogramme, der Workflows und des Gesamtsystems dient. Um die Einzelprogramme möglichst unabhängig von den sie aufrufenden Workflows zu halten, erfolgt die Kommunikation über wohldefinierte Schnittstellen, über welche die Workflows die in „Jobs“ zusammengefassten Einstellungen einspeisen können. Ein Nachteil dieses Ansatzes ist, dass die Komplexität der Abhängigkeiten bei Änderungen an den Schnittstellen oder in den Einzelprogrammen umso mehr steigt, je mehr Workflows sich der betroffenen Programme bedienen – dem soll nach Möglichkeit mit zusätzlichen Abstraktionsebenen begegnet werden, welche Schnittstellendetails nach oben hin verbergen (kapseln).

Die entwickelte Software besteht sowohl aus \textbf{Online- als auch Offline-Teilen}. Die Ausführbarkeit der Programme auf dem lokalen Rechner ist wichtig, um die Hoheit über die eigene Datenverarbeitung (siehe das Grundrecht auf Gewährleistung der Vertraulichkeit und Integrität informationstechnischer Systeme\footnote{\url{https://de.wikipedia.org/wiki/Grundrecht_auf_Gewährleistung_der_Vertraulichkeit_und_Integrität_informationstechnischer_Systeme}}, dessen Verletzung ja nicht fahrlässigerweise begünstigt zu werden braucht) aufrecht erhalten zu können. Ausgenommen davon sind logischerweise Funktionen, die nur in einem Vernetzungs-Kontext Sinn machen wie z.B. eine Benutzerverwaltung, mithilfe welcher die Aufteilung von Aufgaben für kollaboratives Arbeiten an einem Projekt organisiert werden kann. Allerdings können externe Online-Komponenten wie das MediaWiki\footnote{\url{https://www.mediawiki.org/wiki/MediaWiki/de}} oder WordPress\footnote{\url{https://wordpress.org/}} von Online- und/oder Offline-Varianten unserer Programme angesteuert werden, wobei die Unterstützung von dezentralen Peer-to-Peer-Protokollen zwecks Bereitstellung derselben Funktionalität zu bevorzugen wäre. Ob die Einzelprogramme und Workflows mit einem Web-GUI versehen und online ausführbar gemacht werden sollen, hängt vom jeweiligen Nutzungsszenario ab, denn die Verwendung der Online-Version statt der Offline-Version muss im Gegensatz zur Bereitstellung von Diensten für kollaboratives Arbeiten oder für integrierte Plattformen nicht gerade herausgefordert werden.

Ein weiteres wichtiges Anliegen ist die \textbf{Plattform-Integration}. Statt Werke, Werkzeuge, Dienste, Veröffentlichungskanäle und Geschäftsmodelle für sich isoliert zu betrachten, können diese Einzelbausteine miteinander kombiniert werden zu ganzheitlich digitalverträglichen Ökosystemen, in welchen die Produzenten höher entlohnt werden bei gleichzeitig für den Konsumenten günstigeren, quantitativ mehr und qualitativ besseren Ergebnissen. Freilich gibt es keine feste Rollenverteilung in solchen Ökosystemen, jedermann kann gleichberechtigt jede beliebige Funktion den eigenen Fähigkeiten und Fertigkeiten entsprechend ausüben, auch in Kooperation und Kollaboration. Interoperabilität und Vernetzung sind probate Mittel, um die verschiedenen Teilnehmer näher zusammenzubringen. Digitalverträgliche Plattformen fördern die Produktion, Distribution und Rezeption von Werken und erschweren sie nicht künstlich, um Interessen einzelner Parteien zu schützen auf Kosten der ebenso berechtigten Interessen anderer Parteien. Größte Aufmerksamkeit gebührt der Vermeidung von Abhängigkeiten, welche durch Einseitigkeit ein bestehendes Ökosystem ins Ungleichgewicht, in die Instabilität stürzen können. Daher sollten besagte Plattformen flexibel genug konzipiert sein, um einzelne Komponenten austauschen oder den Benutzer vor die Wahl stellen zu können, welche davon eingesetzt werden sollen.

Eine \textbf{Diskriminierung} findet nicht statt. Inhaltliche, personelle, qualitative (nutzungsrechtliche und technische Anforderungen ausgenommen), quantitative, gestalterische, strategische und kommerzielle Präferenzen werden nicht besonders berücksichtigt, sondern sind den handelnden Personen eigenverantwortlich anheim gestellt.

\chapter{Crowdfunding als Entlohnungsmodell digitalverträglicher Ökosysteme}
\section{Erste Umsetzung in der Praxis}
In Ansgar Warners\footnote{\url{http://www.e-book-news.de/}} E-Book-Newsletter vom 23.05.2014 habe ich erfahren, dass in Deutschland der Kladde-Verlag\footnote{\url{http://kladdebuchverlag.de/}} erstmals ausschließlich Crowdfunding\footnote{\url{http://de.wikipedia.org/wiki/Crowdfunding}} für zu realisierende Buchprojekte zugrundelegt. Dabei dient die von Lesern und Investoren vorab bereitgestellte Finanzierung der Deckung der Produktionskosten, sodass der Refinanzierungsdruck herkömmlicher Verlagsmodelle vollständig entfällt. Die Entlohnung des Autors erfolgt über den Verkauf von gedruckten Exemplaren und E-Books, dies aber zu besonders günstigen Konditionen, da die Aufwendungen für die Herstellung bereits abgeglichen sind, sodass bei Fertigstellung eines Projekts vom Verkaufsstart weg die Reingewinn-Spanne angetreten wird. Der Autor kann die freie Lizenzierung der Ergebnisse wählen, sodass die digitalen Grundrechte der Leserschaft gewahrt bleiben und auch andere Bearbeiter, Verlage sowie Distributionskanäle in Betracht kommen können.

Freilich handelt es sich dabei um einen ersten Anfang, ein derartiges Konzept kann aber auch zu einem allgemeinen Marktplatz für Schaffende, Aufbereitende und Leser ausgebaut werden oder gar an eine Distributionsplattform angeschlossen werden. Ferner wären auch andere Autoren-Entlohnungsmodelle auf Basis von Crowdfunding denkbar, sei es pauschal für die Erstellung eines Werkes (Auftragsarbeit), sei es durch den Verkauf einzelner physischer sowie digitaler Exemplare (letzteres als bewusste Unterstützung oder infolge von Bequemlichkeit), sei es durch Subskription eines bestimmten Autors, eines Autoren-Kollektivs, einer Institution, eines Produzenten oder eines Themas, sei es durch Förderung gemeinnütziger Projekte oder über die sonstigen regulär bestehenden Entlohnungsmechanismen.

\section{Ist Crowdfunding überhaupt praktikabel?}
Es wird gesagt, dass Crowdfunding keine besonderen Vorteile aufweisen könnte, mit zunehmender Popularität von Crowdfunding die schiere Anzahl und Vielfalt der Projekte die Crowd überfordern würde, dass nur massentaugliche Literatur umsetzbar wäre und infolgedessen gewagte, unbequeme, abseitige Literatur untergehen würde, dass die Verlagerung des wirtschaftlichen Risikos an die Crowd unseriös sei. Dabei ist es doch schon immer schon so, dass es eher diejenigen Leute gibt, die nur passiv konsumieren und alles möglichst günstig oder gar umsonst haben wollen, und dass es diejenigen gibt, die ein gewisses monetäres Kontingent zur Verfügung haben, welches für kulturelle Bedürfnisse ausgegeben werden kann. Letztere Personengruppe sollte sich gut überlegen, ob es so sinnvoll ist, dieses Geld den restriktiven Rechteverwertern zukommen zu lassen, welche dem Autor nur wenig davon abgeben und nach dem Break-Even\footnote{\url{http://de.wikipedia.org/wiki/Break_even}} bis zu 70 Jahre nach dessen Tod weiter ohne Gegenleistung in die eigene Tasche wirtschaften (unter absichtlicher Beeinträchtigung von tatsächlichen und potentiellen Lesern), oder ob z.B. per Vorabinvestition Autor und Produzenten direkt unterstützt werden und anschließend das Ergebnis auch noch der Allgemeinheit frei zur Verfügung gestellt werden kann.

Es ist schon richtig, dass Crowdfunding in erster Linie nur dann besonders gut funktioniert (in den anderen Fällen kann man mit ein paar Anpassungen nachhelfen), wenn man bereits ein gewisses Publikum hat oder leicht Interesse wecken kann. Allerdings kann ein Autor ganz generell nicht erwarten, Unterstützung von irgendjemandem zu erhalten, wenn besagtes Interesse bei niemandem vorhanden oder zu wecken ist. Man könnte nun sagen, dass herkömmliche Verlage die Investition in inhaltlich weniger attraktiv scheinende Projekte per Quersubventionierung realisieren würden, dazu muss man aber längst nicht als restriktiver Rechteverwerter-Verlag agieren, da ja ausnahmslos jeder, der Geld für eine Quersubventionierung übrig hat, selbiges genauso in Crowdfunding-Projekte investieren kann, ohne die negativen Nebeneffekte des herkömmlichen Verlagsmodells. Dass Crowdfunding nicht besonders populär ist, muss nicht notwendig so bleiben, und die Annahme, dass es mit steigender Popularität weniger funktionieren sollte, halte ich für unbegründet. Die Filterung all der vielen zukünftigen Crowdfunding-Projekte ist ein technisch wie sozial lösbares Problem. Sobald Crowdfunding als akzeptiertes Entlohnungsmodell zunehmende Verbreitung findet, dürfte es auch verstärkt für allerlei Nischen- und Fachthemen zum Einsatz kommen, wo zwar die jeweils interessierte Crowd zahlenmäßig sehr überschaubar ist, dafür aber das Internet die Organisation dieser Interessensgruppen hinreichend vereinfacht hat, sodass die Produktion eigener Veröffentlichungen für solche kleinen, jedoch aktiven Communities methodisch stark vereinfacht wird. Solche Veröffentlichungen können eine große Rolle spielen, wenn sie zur Erhöhung von Relevanz, Reichweite oder Attraktivität des Themas beitragen sollen. Einen wichtigen Beitrag zur Entstehung von Ökosystemen dieser Art können Online-Plattformen und -Präsenzen leisten, die Leserinteresse in Richtung der passenden Projekte kanalisieren oder auch einen freien Markt für allerlei Produktionsdienstleistungen schaffen. Fernerhin käme sogar Mikro-Investment mit niedrigen Funding-Zielen und längeren Laufzeiten (oder wiederholter Neu-Einstellung unter Benachrichtigung aller vormals beteiligten Interessenten) nach dem Vorbild vieler Bibliotheken in Betracht, die bisher jede für sich isoliert die gemeinnützige Digitalisierung und Restauration alter Bücher durch mehr oder minder zufällige Besucher ihrer Webseiten zu finanzieren suchen. Ebenso kann auch die Arbeit an frei lizenzierten, digital vorliegenden Werken iterativ gestaltet werden, wenn einzelne Verbesserungen durch einzelne Interessenten veranlasst werden und somit ein gemeinsamer Fortschritt erzielt wird.

Crowdfunding ist auch kein Outsourcing des Risikos, stattdessen wird das Risiko gänzlich ausgeschaltet. Für den Autor, weil er gar nicht erst in ein Projekt investieren muss, für das es gar kein Publikum gibt. Für die Produzenten, weil sie nicht tätig werden müssen, bevor ihre Leistung durch die Crowd vollständig kompensiert werden kann. Für die Crowd, weil sie die getätigten Investitionen zurückerhält, wenn der benötigte Betrag nicht zustande kommt. Ein Verlag wirkt da schon eher wie ein Risiko-Spekulant, der bei Fehlkalkulation auch schnell mal Insolvenz anmelden kann (mit entsprechenden Folgen für Mitarbeiter, Autoren und die Kunden im Markt).

\section{Verhältnis von Crowdfunding zur Kunst}
Schreiben als kreativer Prozess unterscheidet ich von der Produktion materieller Güter. Bei letzterer können Prototypen als Entscheidungshilfe für die Crowd dienen, ob ein Projekt gefällt oder nicht. Bei Büchern finden viele Titel ihr Publikum nach der Veröffentlichung, während des Schreibens ändert der Autor die Handlung, der potentielle Leser könnte im Rahmen der Projektvorstellung den Inhalt bereits kennen lernen und somit den Anlass für die Investition verlieren, da das Ergebnis nichts neues oder überraschendes mehr zu bieten hätte. In der bildenden Kunst kann auch nicht im Voraus für ein Gemälde oder eine Skulptur bezahlt werden, welche dem Künstler gerade mal vor dem inneren Auge vorschwebt. Auch Erstlingswerke werden es mit Crowdfunding schwer haben, etablierte Autoren dagegen leichter.

Im Fall des Kladde-Verlags reichen Autoren wie sonst überall auch Manuskripte ein, die ganz normal geprüft werden (zusätzlich auch auf Crowdfunding-Tauglichkeit). Wenn jemand kein Manuskript hat, keine Anhängerschaft und keine attraktive Idee (letztere ist bekanntlich am allerwenigsten wert), dann besteht doch völlig unabhängig vom Kompensationsmodell generell keine Chance auf die Unterstützung einer Umsetzung, wenn man mal von Auftragsarbeit absieht. Irgend etwas muss der Autor schon von sich aus tun, und wenn das, was er tut, gut ist, kann er auch Unterstützer finden, sei es auf die eine oder andere Weise und unter Berücksichtigung des Wesens des angestrebten Werkes.

Wie beim Gemälde oder der Skulptur, wenn der Maler oder Bildhauer noch nie etwas geschaffen, seine künstlerischen Fähigkeiten unter Beweis gestellt hat, dann wird er auch von niemandem einfach so Geld dafür erhalten. Es wird an kein Finanzierungsmodell jemals der Anspruch gestellt, so etwas leisten zu müssen, also auch an Crowdfunding nicht.

Bei der Kunst geht es aber auch viel um die konkrete Ausgestaltung, um den im Kunstwerk festgehaltenen Ausdruck des Künstlers. Von daher ist es sehr wohl denkbar, die abstrakte Idee in all ihren Details den potentiellen Crowdfunding-Investoren vorzustellen, während gleichzeitig der Wert des Ergebnisses in den vielen Einzelentscheidungen des Künstlers im Rahmen der Umsetzung liegt, welche die Crowd weder abnehmen noch mitbestimmen kann. Für Werke der Unterhaltung trifft dies natürlich weit weniger zu, die Produktion derselben folgt wiederum anderen Regeln, wo die Eignung zum Crowdfunding z.B. durch Mitgestaltung, Serien-Fortsetzungen und gemeinsame Rezeption gegeben ist.

\section{Fazit}
Der Crowdfunding-Ansatz ist in der Lage, die Umsetzung frei lizenzierter Projekte erheblich zu vereinfachen und damit die ganz zentralen Nachteile des herkömmlichen Geschäftsmodells auf Basis der restriktiven Rechteverwertung aufzuheben. Wird deshalb die restriktive Rechteverwertung aufgegeben werden? Sicher (noch) nicht. Ist dadurch das Crowdfunding-Modell unbrauchbar? Natürlich ebenfalls nicht. Und sowie diese Alternative in der Praxis unter günstigeren Konditionen für alle Beteiligten mehr und qualitativ bessere Projekte realisieren wird, desto mehr bröckelt die Relevanz des alten Modells. Solange nur beide Optionen zur Auswahl stehen, ist auch ziemlich egal, welche davon mehr in Anspruch genommen wird, da es immer Menschen geben wird, die sich an den ganzheitlich geeigneteren Ansatz halten werden. Dies hängt auch ganz individuell davon ab, ob die eigene Zielsetzung darauf hinausläuft, einen künstlichen Exklusivismus aufbauen und vehement verteidigen zu wollen, oder ob der Wille zum Beitrag für die Gemeinschaft überwiegt. Mag sein, dass ein kooperativer Ansatz für den einen oder anderen nichts ist, für den ist aber auch Digitaltechnologie und das Internet\footnote{\url{http://www.youtube.com/watch?v=kK1wtmyFll4}} nichts.

\end{document}
